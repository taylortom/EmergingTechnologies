\documentclass[runningheads,a4paper]{llncs}

\usepackage{amssymb}

\usepackage{url}
\urldef{\mail}\path|tt36@uni.brighton.ac.uk|
\newcommand{\keywords}[1]{\par\addvspace\baselineskip
\noindent\keywordname\enspace\ignorespaces#1}

\begin{document}

\mainmatter  

\title{The Use of Quantum Mechanical Effects to Provide Greater Security in Data Communication}
\titlerunning{Quantum Mechanical Effects and Secure Data Communication}
\author{Thomas Taylor\\ \mail}
\authorrunning{Quantum Mechanicals and Secure Data Communication} % (feature abused to repeat the title on left hand pages)
\institute{School of Computing, Engineering and Mathematics,\\University of Brighton.}

\maketitle

\begin{abstract}
A considerable weakness of current cryptographic techniques lies in the fact that they rely upon the inability for current generation hardware to factor large numbers in polynomial time. With the discovery of a quantum algorithm capable of solving such a problem, we discuss how quantum effects can be utilised to provide an alternative and more secure method of cryptography.

\end{abstract}

\section{Introduction}

The primary concern when it comes to accessing sensitive data is \emph{privacy}. More specifically, how to enable secure, private communication between two parties such that any intermediaries listening in to the transmission are unable to obtain any useable information from it\cite{Rivest:1990fk}. Given the increasing amount of sensitive data accessible over the internet such as medical records, personal banking and employee payrolls, in addition to the number of social networks and other online services which store individuals' personal data, cryptography and data security has never been more important.

With data security of such paramount importance, it is surprising to see that the classical\footnote[1]{The term 'classical' is used in this paper to refer to computer and cryptographic systems which do not utilise quantum mechanical effects (i.e. all modern systems used today).} cryptographic systems we use today still rely on computational complexity as their best line of defence. Breaking such techniques becomes a fairly trivial matter given enough time and computing power. Whilst this does not prove problematic with today's hardware, with the recent developments in computer systems capable of quantum computation\cite{Lu:2007uq}, as well as the existence of quantum algorithms capable of providing polynomial speed up of integer factorisation\cite{Shor:1994fk}, modern cryptographic techniques are looking ever more vulnerable to attack. The need for alternative forms of cryptography which can offer better long-term security is becoming increasingly apparent.

\section{Quantum Mechanical Concepts}

There are a number of key quantum mechanical principles which are utilised in quantum cryptography to provide a greater level of security than possible using classical techniques. Brief explanations of quantum mechanical concepts relevant to the later sections are given here.

\subsection{Quantum bits and Quantum States}

In classical computing, binary digits (or bits) form the basic building blocks of data. The quantum bit (or qubit) is the quantum computing equivalent, acting both as a processor as well as memory. 

While in classical computing, a bit is capable of existing in one of only two states (0 and 1), the qubit can also exist in an additional 'superposition' state. ****(the superposition state is essentially an infinite number of in-between states)****. Additionally, whereas a classical bit is capable of existing in only one state at any one time, before a qubit is measured, ****it is said to exist in all of its states at once****. However, as soon as the qubit is measured, it transitions to a single state. The fact that a qubit changes its state when measured is a very useful behaviour, and is utilised very effectively in QKD, which is described later in section 5.

\subsection{Quantum entanglement}

Quantum entanglement is another unique quantum mechanical behaviour observed when multiple qubits interact with one another. When coming into contact, the qubits are said to have become 'entangled'. This interaction causes the two qubits to become associated with each other in such a way that if a measurement of either of the pair is made (with its state being altered as a result), the state of the other qubit is also changed to the complementary state of its partner. This behaviour has been shown to occur irrespective of the distance between the qubits**[citation]**.

\section{Asymmetric Key Cryptography (Public-Key)} 

The asymmetric (or public-key) method is the most popular cryptographic technique used today, and underpins many integral internet standards such as Transport Layer Security (TLS), GNU Privacy Guard (GPG), and Pretty Good Privacy (PGP).

One of the main benefits to using the public-key system is that whilst the generation of the keys and the encryption/decryption processes are trivial for the sender and receiver, it is incredibly expensive computationally for an attacker to decipher the message. The use of public keys also removes the need for a secure private exchange of keys between participants, which is a requirement for some other cryptographic methods (such as the symmetric key method).

The security of public-key cryptography lies in the fact that it is currently incredibly computationally expensive (and therefore infeasible) to factor very large numbers, or at least to do so in a useable timeframe. However, this could also be viewed as a disadvantage, given the fact that any public-key algorithm can be broken fairly trivially provided that time and computational power are not a limiting factor.

\section{Integer Factorisation}

In classical computing, the most efficient algorithm currently available for factoring numbers with more than 100 digits is the general number field sieve (GNFS). The GNFS algorithm has a running time which while sub-exponential, is still super-polynomial in the size of the input. To give an idea of what this means in real terms, researchers concluded an experiment in December 2009 into the breaking of an RSA-768 number (consisting of 232 digits) using the GNFS algorithm. It took the team a total of 2 years to factor the number, with their estimation being that breaking an RSA-1024 number would be a thousand times harder\cite{Kleinjung:2010:FRM:1881412.1881436}. 

Although these findings could easily lead one to believe that the security of the public-key method is assured for some time to come, the fact that their strength relies on computational complexity means that an unforeseen advancement in  computer hardware could severely compromise its security.

This vulnerability of public-key based systems was further amplified in 1994, when Peter Shor discovered a quantum algorithm which was capable of factoring integers in polynomial time, providing an exponential speed-up over the general number field sieve\cite{Shor:1994fk}. 

With this discovery came the realisation that as soon as sufficiently powerful systems capable of quantum computation became available, attackers would be able to efficiently break into public-key systems, effectively rendering them completely insecure.

Shor's discovery was further substantiated in 2001 when researchers at IBM demonstrated the algorithm on a practical quantum computer by factoring 15 using 7 qubits\cite{Vandersypen:2001fk}. Although this demonstration was disregarded by some due to the fact that no quantum entanglement had been observed, subsequent experiments have since been carried out which support IBM's findings\cite{Lu:2007uq}.

****Although the development of physical quantum computers is still in its infancy, the fact that such computation has been proven both theoretically, and in practical demonstrations is extremely dangerous to the security of our sensitive data.****

\section{Quantum key distribution}

With the discovery of Shor's quantum algorithm finally rendering large-number integer factorisation a feasible possibility, there was a need to develop new cryptographic techniques which could utilise the same quantum mechanical concepts in order to provide more secure data transmission. QKD provides a solution to this problem. QKD functions in a similar way to public-key cryptography, but makes use of quantum effects to ensure much greater privacy.

As with the public-key method, the first step of QKD is for the sender to encrypt the intended message using a secret key which is generated from random quantum states. The message is then sent to the intended recipient. Upon receipt of the transmission, the recipient measures the data using his own quantum-state-key. The two parties then publicly communicate to each other the method that they used to prepare (or measure) the transmission. From this, they are able ascertain which bits of the transmission were prepared in the same way, and collate these bits into a new string. Provided there was no interference during the transmission, these two strings should be identical to one another. However, in practice, a transmission is highly unlikely to be completely error-free, and so is considered to be secure if it has an error rate of less than 25\%\cite{Steane:1997zr}. Due to the fact that to any gain information from the transmission any intermediaries must measure the transmitted data (thus causing the state of any measured qubits to change as a result),  their presence will be easily detectable to the two communicating parties.

One of the biggest advantages to using QKD over existing public-key systems is the fact that the transmission of data is protected by the laws of quantum physics rather than computation complexity, meaning that its security will be ensured regardless of any unforeseen technological advances in the future. In addition, as mentioned, QKD systems are able to effectively utilise quantum entanglement in order to allow the two communicating parties to detect the presence of any third parties; something not possible with existing classical methods. Another key benefit of QKD is that it still allows public channels to be used to transmit the data, which is a much less costly solution than having to communicate via private channels. This also opens up the opportunity to use QKD as a secure method of communication for surface-to-satellite, satellite-to-satellite or potentially even deep-space\cite{Hughes:2000uq}.

With various rigorous mathematical proofs having been established which confirm the security of QKD\cite{Deutsch:1996fk}\cite{Shor:2000uq}, it clearly offers a promising new method of cryptography. In fact, there are already several companies which are currently offering commercial QKD systems. However, major flaws have been found in these commercial systems which mean that a third party is able to manipulate quantum mechanical effects to eavesdrop on the transmission undetected\cite{Wiechers:2011fk}\cite{Zhao:2008fk}. For example, researchers at the University of Toronto led by Yi Zhao have recently shown that it is not only possible, but technologically feasible using \emph{current} technology to mount an undetected attack against a commercial QKD system. Known as a time-shift attack, an eavesdropper is able exploit the efficiency of detectors used to receive the data transmission and gain access to its information content. Most surprising about Zhao et al's discovery is that their proposed attacker was much weaker than the one proposed in the original security proofs. The attacker proposed in the paper was unable to 'perform a quantum non-demolition (QND) measurement on the photon number or compensate any loss introduced by the attack', while the attacker in the original proofs could have 'arbitrarily advanced technology'. Using this method, the eavesdropper might be able to ascertain the whole key, or at least enough of it for a brute force attack to prove successful. 

Although these commercial solutions have been proven to be insecure, it is important to note that the systems in fact 'deviate from the models in the security proofs'\cite{Lydersen:2010qy}, and so do not invalidate the underlying concept of QKD. 

\section{Conclusion}

With so much sensitive data accessible over the internet, it is critical that the methods we use to transmit that data are suitably secure. While the public-key systems that we use today are suitably secure against attacks from current generation hardware, they cannot provide the same kind of guarantee against the quantum-based hardware which is currently being developed.

QKD provides a promising alternative to the existing public-key method. In addition to relying on the laws of quantum physics for security rather than computational complexity, QKD also allows the communicating parties to detect the presence of any third parties; a key benefit over classical cryptographic methods.

However, although QKD has been proven secure in theory, there have yet to be any practical solutions which demonstrate the same kind of security. 

\bibliographystyle{abbrv}
\bibliography{EmergingTechnologies}

\end{document}
