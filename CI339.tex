\documentclass[runningheads,a4paper]{llncs}

\usepackage{amssymb}

\usepackage{url}
\urldef{\mail}\path|tt36@uni.brighton.ac.uk|
\newcommand{\keywords}[1]{\par\addvspace\baselineskip
\noindent\keywordname\enspace\ignorespaces#1}

\begin{document}

\mainmatter  

\title{The Use of Quantum Mechanical Effects to Provide Greater Security in Data Communication}
\titlerunning{Quantum Mechanical Effects and Secure Data Communication}
\author{Thomas Taylor\\ \mail}
\authorrunning{Quantum Mechanical Effects and Secure Data Communication} % (feature abused to repeat the title on left hand pages)
\institute{School of Computing, Engineering and Mathematics,\\University of Brighton.}

\maketitle

\begin{abstract}
The biggest weakness of current cryptographic techniques is that they rely upon the inability for current generation hardware to factor large numbers in a useable time-frame. With recent advances in making physical quantum computers a reality, it's becoming increasingly likely that we will soon have hardware capable of applying quantum mechanical algorithms in order to break classical cryptographic techniques.
\end{abstract}

\section{Introduction}

The primary concern when it comes to accessing sensitive data is \emph{privacy}; more specifically, how to enable secure, private communication between two parties such that any adversaries that may be listening in know nothing about the data that has been transmitted\cite{Rivest:1990fk}. Given the increasing amount of sensitive data accessible over the internet such as medical records, personal banking and employee payrolls, not to mention the number of social networks and other online services which store individuals' personal data, cryptography and data security has never been more important.

It is somewhat surprising then that the classical cryptographic techniques that we use today are still largely reliant on the performance limitations of current generation hardware; such techniques can be easily broken given enough time and computing power. Whilst this does not prove problematic with today's hardware, the recent developments in computer systems capable of quantum computation, and the existence of quantum algorithms capable of providing polynomial speed up of integer factorisation\cite{Shor:1994fk}, modern cryptographic techniques are looking ever more vulnerable to attack, and the need for more secure forms of cryptography are becoming increasingly apparent.

\section{Quantum Mechanical Concepts}

There are a number of key quantum mechanical principles which are utilised in quantum cryptography and specifically in quantum key distribution (QKD). I have given a brief explanation of relevant concepts in this section.

\subsection{Quantum bits and Quantum States}

In classical computing, binary digits (or bits) form the basic building block of data. The quantum bit (or qubit) is the quantum computing equivalent, acting both as a processor as well as memory. 

While in classical computing, a bit is capable of existing in one of only two states (0 and 1), in quantum computing, the qubit can exist in a 0 or 1 state, or as a superposition of both these two states (the superposition state is essentially an infinite number of in-between states). Before a qubit is measured, it is said to exist in all of its states at once (unlike a classical bit, which is capable of existing in only one state at any one time). However, as soon as the qubit is measured, it returns a single state.

Qubits exhibit a very unique behaviour when they are measured: when a measurement is made, the qubit changes its state. This behaviour is utilised very effectively in QKD, which is described later.

\subsection{Quantum entanglement}

Quantum entanglement is another unique quantum mechanical behaviour observed when multiple qubits interact with one another. When coming into contact, the qubits are said to have become 'entangled'. Just as individual qubits have a 'superposition' state, so too do the entangled qubits, which can be described as a superposition of the superposition state of the qubits.

This interaction causes the two qubits to become associated with each other in such a way that if a measurement of either of the pair is made (with its state being altered as a result), the state of the other qubit is also changed to the complementary state of its partner. This behaviour occurs irrespective of the distance between the qubits.

\section{Integer Factorisation}

In classical computing, the most efficient algorithm currently available for factoring numbers with more than 100 digits is the General Number Field Sieve whose running time, while sub-exponential, is still super-polynomial in the size of the input. In December 2009, researchers concluded an experiment into the breaking of an RSA-768 number of 232 digits, using the number field sieve algorithm. It took the team a total of 2 years to factor the number, with their estimation being that breaking an RSA-1024 number would be a thousand times harder\cite{Kleinjung:2010:FRM:1881412.1881436}. Although this method has so far proven secure for obvious reasons and using larger numbers would provide greater security, this is not a preferable solution due to the fact that the security of the public key system relies on inadequacies of modern computer systems. 

This weakness was amplified in 1994, when Peter Shor discovered a quantum algorithm which was capable of factoring integers in polynomial time\cite{Shor:1994fk}, and provided an exponential speed-up over the general number field sieve. Suddenly the public-key cryptographic techniques which had become a standard looked vulnerable to attack. Although the development of quantum computers capable of such powerful computation is still in its infancy,   the fact that such computation is theoretically possible is extremely dangerous to the integrity of our sensitive data.

This was further proven in 2001 when IBM demonstrated the algorithm on a practical quantum computer by factoring 15 using 7 qubits\cite{Vandersypen:2001fk}. Although this demonstration was disregarded by some as no quantum entanglement had been observed, subsequent experiments have since been carried out which support IBM's findings\cite{Lu:2007uq}.

\section{Asymmetric Key Cryptography (Public Key)} 

The asymmetric (or public key) method is the most popular cryptographic technique used today, and underpins many integral internet standards such as Transport Layer Security (TLS), GNU Privacy Guard (GPG), and Pretty Good Privacy (PGP).

One of the main benefits to using the public key system is that the generation of the keys and the encryption/decryption processes are trivial for the sender and receiver, whereas it is incredibly expensive computationally for an attacker to decypher the message. Public keys also remove the need for a secure exchange of keys between participants, which is a requirement for some other cryptographic methods (such as the symmetric key method).

The security of public key cryptography lies in the fact that it is currently incredibly computationally expensive (and therefore infeasible) to factor very large numbers, or at least to do so in a useable timeframe. However, this could also be viewed as a disadvantage, given the fact that any public key algorithm can be broken fairly trivially (provided that the time and computational power available are not a limiting factor).

\section{Quantum key distribution}

With the discovery of Shor's quantum algorithm suddenly rendering the public key method insecure (in theory at least), there was a need to develop new cryptographic techniques which could utilise the same quantum mechanical concepts in order to provide more secure data transmission. QKD provides a solution to this problem.

QKD functions in a very similar way to public key cryptography, but makes use of quantum effects to ensure much greater privacy. Initially, the intended message is encrypted by the sender 'Alice' using a secret key generated from a random quantum state. Upon receipt of the transmission, the receiver 'Bob' measures the message using his own quantum-state-key. The two parties then let each other know the method they used to prepare (or measure) the transmission. From this, they can ascertain which bits were prepared in the same way, and collate these bits into a new string. Provided there was no interference, these two strings should be identical to one another, although in practise, a transmission is considered secure if it has an error rate of considerably less than 25\%\cite{Steane:1997zr}. Any eavesdroppers attempting to gain access to the transmission will make their evidence known to Alice and Bob due to the fact that merely measuring the state of the transmission causes it to change its state, thus increasing the error rate.

The biggest advantage to using QKD over existing public key systems is the fact that the data transmission is protected by the laws of quantum physics, as opposed to public key methods, whereby computational complexity and the performance of current systems mean that the security is guaranteed. In addition, QKD systems are able to detect whether the transmitted message has been compromised by a third party, something not possible with the public key method.

Although rigorous mathematical proofs have been established which seemingly confirm the security of QKD\cite{Deutsch:1996fk}\cite{Shor:2000uq}, researchers at the University of Toronto have recently shown that it is not only possible, but technologically feasible using \emph{current} technology to mount an undetected attack against a \emph{commercial} QKD system. Known as a time-shift attack, an eavesdropper 'Eve' can exploit the efficiency of detectors used to receive the transmission to break the security of the system\cite{Zhao:2008fk}. Although these commercial solutions have been proven insecure, it is important to note that the commercial QKD systems 'deviate from the models in the security proofs'\cite{Lydersen:2010qy}. Most surprising about this discovery though, is that Zhao et al's proposed attacker was much weaker than the one proposed in the original security proofs; the attacker proposed in the paper was unable to 'perform a quantum non-demolition (QND) measurement on the photon number or compensate any loss introduced by the attack', while the attacker in the proofs could have 'arbitrarily advanced technology'. Using this method, Eve might be able to ascertain the whole key, or at least to obtain enough of the key for a brute force attack to prove successful.

\section{Conclusion}

Theoretically, QKD provides a completely secure method of data transmission the type of which is simply not possible using classical cryptographic methods (or even classical computing). Furthermore, the security provided by QKD is not reliant on computational complexity unlike classical methods, and therefore provides a considerably more 'future-proof' solution. 

However, while the theory behind QKD has been rigorously tested on paper, there have yet to be any viable implementations of such a system. This, combined with the fact that we have yet to see a quantum computer system developed which is capable of performing functions even remotely close to the complexity of those of current generation hardware, such a cryptographic system still seems likely to be a way off.

\bibliographystyle{abbrv}
\bibliography{EmergingTechnologies}

\end{document}
